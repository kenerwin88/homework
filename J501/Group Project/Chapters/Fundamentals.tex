%************************************************
\chapter{About Dorel Industries}
\label{chp:about}
%************************************************

In 1962, Leo Schwartz founded "Dorel Co. Ltd" in Quebec, and began producing juvenile products.  It was not until 1962 that "Dorel Co. LTD" became "Dorel Industries", following a merger with Ridgewood Industries (a furniture manufacturing company).  Since then, the company has continued to grow primarily through acquisitions, eventually branching out to the recreational and leisure markets by acquiring Schwinn and Cannondale


\section{Recent Strategic Moves (2013)}
\begin{itemize}
  \item Dorel began a massive share buyback plan in order to raise its market value.
  \item The company acquired a 70% controlling interest in Caloi (a major brazilian bicycle manufacturer).  This acquisition is intended to help Dorel expand into Latin America, as Caloi is the number one bicycle brand in the region.
  \item Dorel’s assembly and testing facilities located in Bedford, PA are being shut down and relocated overseas in an effort to reduce expenses.
\end{itemize}


%************************************************
\chapter{Business Overview}
\label{chp:overview}
%************************************************


\section{Business Definition}

Dorel Industries has a diverse business definition.   Dorel makes high quality furniture, juvenile and recreational products for consumers who emphasize quality and durable products.  Dorel’s technologies include ready-to-assemble furniture, high quality and durable bicycles as well as safe and reliable baby products.

As you can see in the below figure, Dorel’s business is comprised of very distinct products amongst distinct markets. Dorel’s main products include furniture, bicycles and baby products.  These products relate to Dorel’s business units.   In term’s of Dorel’s target markets, United States is a very important market. However, due to the US economy downturn, this has made Dorel focus on extending their business internationally.  Lastly, the technology that Dorel institutes is one of outsourcing manufacture of their products, in order to be able to compete on price, specifically for their home furnishings business unit.

One, of many, issues with this business definition is there does not appear to be synergies amongst either the manufacture of the products, the products themselves, nor the target markets of each product. As part of our recommendation, Dorel would benefit from finding a way to have more synergies amongst their most important products. 

\begin{itemize}        	
  \item Products
    \begin{itemize}
      \item Furniture
      \item Bicycles
      \item Baby products/accessories
    \end{itemize}
  \item Markets (respectively)
    \begin{itemize}     
      \item North American retail chains
      \item Mass merchant / Independent Bike Dealer (IBD) network
      \item US and International retail chains
    \end{itemize}
  \item Technology
    \begin{itemize}     
      \item Ready-to-assemble furniture
      \item High Quality products
      \item Safe and durable juvenile products
    \end{itemize}
\end{itemize}


\section{Business Unit Breakdown}
Dore’s business is comprised of 3 distinct business units; Juvenile, Home Furnishings and Recreational/Leisure.  These 3 business units drove 2012 revenue of \$2.49 Billion, as well as \$583 Million in gross profit. A breakdown of each business unit follows.

\subsection{Juvenile}
The Juvenile business unit is focused on the manufacture and import of high quality, safe and fashionable juvenile products.  These products include car seats, strollers, high chairs, etc.  Products are provided under their own brand names, as well as house brand names for their customers. This segment produced 2012 revenues of \$1.04 Million, which equates to 42% of the business. This segment produced \$287 Million in gross profit which was 49% of Dorel’s total gross profit.

Dorel is focusing on growing this segment in Latin America, where the retail environment is beginning to prosper and birthrates in this region are an incline. Dorel has also made very recent acquisitions for this segment to expand its breadth of offering as well as introduction to new international channels.

\subsection{Recreational/Leisure}
The Recreation/Leisure business is comprised of premium/mass market bicycles, jogging strollers, ride-on toys as well as branded performance apparel.  This segment has a focus on international markets, with 50% of sales coming from the Asia-Pacific region (US accounts or 12%). This segment accounted for 37% of Dorel’s 2012 revenue, or \$928 Million, with gross profit at \$233 Million.

Dorel is focused on making this segment the premier bicycle business in the market. in 2012, selling expenses for this business unit increased 13%, which leadership attributed to marketing efforts to enrich this segment’s brands.

\subsection{Home Furnishings}
The Home Furnishings business focuses on ready-to-assemble furniture, step stools, futons and imported home entertainment furniture.  The primary focus for this segment is North American markets, which is evident since Dorel has five distinct segments within this business unit. This segment drove 21 percent of Dorel’s 2012 revenue with \$521 million, as well as 11% of gross profit with \$62 million

This segment is the cash cow for Dorel.  However, with this segment’s focus being on the North American market, home-related market, this segment has suffered a bit with the US economy.  Recenty, Dorel moved more of its manufacturing overseas, which should help with this segment’s ability to compete on price in their mass retail chains.

\begin{table}[h]
    \begin{tabular}{lllllllllll}
    {\bf \underline{Unit}}                 & {\bf \underline{Total Sales}}    & {\bf \underline{\%}}  & {\bf \underline{Gross Profit}} & {\bf \underline{\%}}\\
    Juvenile             & \$1,040,765,000 & 42\% & \$287,658,000 & 49\% \\
    Home Furnishings     & \$521,523,000  & 21\% & \$62,552,000 & 11\% \\
    Recreational/Leisure & \$928,422,000  & 37\% & \$233,437,000 & 40\% \\
    \end{tabular}
\end{table}

\begin{table}[h]
    \begin{tabular}{lllllllllll}
    {\bf \underline{Unit}}             &  {\bf \underline{Operating Profit}} & {\bf \underline{\%}}  & {\bf \underline{Unit Strength}} \\
    Juvenile             & \$73,313,000     & 43\% & High \\
    Home Furnishings     & \$25,593,000     & 15\% & Very Low \\
    Recreational/Leisure  & \$71,958,000     & 42\% & High \\
    \end{tabular}
\end{table}

\begin{table}[h]
    \begin{tabular}{lllllllllll}
    {\bf \underline{Unit}}               &{\bf \underline{Unit Rating}} &  {\bf \underline{Industry Potential}} &{\bf \underline{Industry Rating}} \\
    Juvenile             & 3           & 5                  & 2               \\
    Home Furnishings      & 1           & 3                  & 2               \\
    Recreational/Leisure  & 4           & 5                  & 4               \\
    \end{tabular}
\end{table}
